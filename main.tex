%----------------------------------------------------------------------------------------
%	PACKAGES AND DOCUMENT CONFIGURATIONS
%----------------------------------------------------------------------------------------
% !TEX root=main.tex
\documentclass[10pt]{IEEEtran}

\usepackage[utf8]{inputenc}
\usepackage{graphicx}
\usepackage[english,spanish,activeacute]{babel}
\usepackage[cmex10]{amsmath}
\usepackage{hyperref}
\hypersetup{
    colorlinks=true,
    linkcolor=blue,
    urlcolor=blue,
    citecolor=black
}

\usepackage[numbers]{natbib}
\usepackage{xurl}
\usepackage[acronym]{glossaries}
\usepackage{float}

%----------------------------------------------------------------------------------------
% Acronyms
%----------------------------------------------------------------------------------------
\newacronym{pcr}{PCR}{Polymerase Chain Reaction}
\newacronym{tgf}{TGF}{Temperature Gradient Focusing}




\begin{document}

\newcommand{\titlepaper}{Least-Squares Filter Applied to a Temperature Control System using Simulink}

\title{\titlepaper}

\selectlanguage{english}
\author{\IEEEauthorblockN{Erick Andrés Obregón Fonseca}

\IEEEauthorblockA{erickof@estudiantec.cr}
\IEEEauthorblockA{\\MSc in Electronics -- Microelectronics Emphasis}
\IEEEauthorblockA{\\Instituto Tecnológico de Costa Rica}
}


\markboth{Obregón-Fonseca, \titlepaper}
{Shell \MakeLowercase{\textit{et al.}}: Bare Demo of IEEEtran.cls for Journals}


\IEEEtitleabstractindextext{%

\renewcommand{\abstractname}{Abstract}

\begin{abstract}
Accurate temperature monitoring is crucial in numerous daily and industrial applications. To address this, the utilization of a least-squares filter to effectively mitigate noise in the measured variable is explored in this paper.

For the study, a heating system was simulated using Simulink, incorporating Gaussian white noise into the output to mirror real-world disturbances. The system maintains a set-point of 25$^\circ$C, which represents the ambient temperature, and features an input pulse that activates the heating mechanism.

The findings indicate that the least-squares estimation technique successfully approximates the actual temperature with a high degree of accuracy, achieving results that closely match the true values.
\end{abstract}


\renewcommand{\IEEEkeywordsname}{Keywords}

\begin{IEEEkeywords}
Control System, Least Squares Filter, Matlab, Simulink, Temperature
\end{IEEEkeywords}}

\maketitle

\IEEEdisplaynontitleabstractindextext

\IEEEpeerreviewmaketitle

\selectlanguage{english}

\section{Introduction}

\subsection{Heating and Temperature Control}
Heating and temperature control systems have played a crucial role since the end of the 19th century. These systems have a wide range of applications from day-to-day tasks to industrial applications. \\

In 1895, Johnson made a significant breakthrough in temperature control with an automatic multi-zone temperature control system, designed to regulate the temperature in individual rooms or apartments~\cite{us542733s}. Subsequentially, a variety of other applications have emerged. In the field of biotechnology, applications like~\acrfull{pcr}--such~\cite{MULLIS1987335, saiki1988, Bartlett2003, C6LC00984K, maltezos2010, mcknight2000, B208405H, hua2010multiplexed, mahjoob2008rapid, dinca2009fast, lien2009microfluidic, qiu2010large, hsieh2008enhancement, shen2005microchip, wang2009miniaturized}--and~\acrfull{tgf}~\cite{matsui2007temperature, ross2002microfluidic} for Electrophoresis requires tight temperature control~\cite{diagnostics3010033}. In biological science, precise temperature control is crucial, such as maintaining specific temperatures for cell viability or for~\acrshort{pcr} temperature cycling~\cite{diagnostics3010033, hung2005microfluid}. Also, in the food supply chain is important to ensure product quality and customer satisfaction, in transportation modes like sea, land, and railway, and in the logistic clusterization process can significantly reduce costs~\cite{baskutis2015temp}. On the other hand, in the automotive industry, thermal management in vehicle electrification enhances vehicle efficiency and battery performance~\cite{casals2016sustainability, previati2022thermal} and in specific areas like thermal management of lithium-ion batteries~\cite{karimi2013thermal}, electric machines ~\cite{yang2017thermal}, and novel thermal management systems for batteries~\cite{al2018review}. \\

In critical applications such as mentioned before, accurate temperature monitoring is imperative. Consequently, the utilization of a least-squares filter to effectively mitigate noise in the target variable is explored.


\subsection{Least-Squares Filter}
Least-squares estimation theory was introduced by Gauss and further developed by Kalman, focusing on minimizing the sum of the squares of the difference between observed and estimated values~\cite{sorenson1970lse}. This methodology finds applications across a broad spectrum of categories, such as data curve fitting, parameter identification, and the realization of system models~\cite{crassidis2004dynamic}. This technique is versatile, with applications spanning various domains. Examples include calculating the damping properties of a fluid-filled damper based on temperature, identifying aircraft dynamic and static aerodynamic coefficients, determining orbit and attitude, locating position using triangulation, identifying modes of vibratory systems, and modern control strategies like some adaptative controllers where least-squares method is used to refine model parameters within the control system~\cite{crassidis2004dynamic}. \\

Assuming a set or a batch of measured values, $\tilde{y}_{j}$, of a process $y(t)$, taken at known discrete instants of time $t_{j}$:

\begin{equation}
    \left\{\tilde{y}_{1}, t_{1}; \tilde{y}_{2}, t_{2}; \ldots; \tilde{y}_{m}, t_{m}; \right\}
\end{equation}

and a proposed mathematical model of the form

\begin{equation}
    y(t) = \sum_{i = 1}^{n}{x_{i} h_{i}(t)},~~~m \geq n
\end{equation}

where

\begin{equation}
    h_{i}(t) \in \left\{ h_{1}(t), h_{2}(t), \ldots, h_{n}(t) \right\}
\end{equation}

are a set of independently specified basis functions where the measurements $\tilde{y}_{j}$ and the estimated output $\hat{y}_{j}$ can be related to the true and the estimated x-values leading to the Eq~\ref{eq:lse_w_error} and~\ref{eq:lse_wo_error} ~\cite{crassidis2004dynamic}:

\begin{equation}
    \tilde{y}_{j} \equiv \tilde{y}(t_{j}) = \sum_{i = 1}^{n}{x_{i} h_{i}(t_{j}) + v_{j}},~~~j = 1, 2, \ldots, m
    \label{eq:lse_w_error}
\end{equation}

\begin{equation}
    \hat{y}_{j} \equiv \hat{y}(t_{j}) = \sum_{i = 1}^{n}{\hat{x}_{i} h_{i}(t_{j})},~~~j = 1, 2, \ldots, m
    \label{eq:lse_wo_error}
\end{equation}

where $v_{j}$ is the measurement error. This leads to the following identity:

\begin{equation}
    \tilde{y}_{j} = \sum_{i = 1}^{n}{\hat{x}_{i} h_{i}(t_{j}) + e_{j}},~~~j = 1, 2, \ldots, m
    \label{eq:lse_w_rerror}
\end{equation}

where residual error $e_{j}$ is defined by

\begin{equation}
    e_{j} \equiv \tilde{y}_{j} - \hat{y}_{j}
    \label{eq:residual_error}
\end{equation}

and this can be rewritten in compact matrix form as:

\begin{equation}
    \bf{\tilde{y}} = \bf{\it{H}} \bf{\hat{x}} + \bf{e}
\end{equation}

where ~\cite{crassidis2004dynamic} \\

$\bf{\tilde{y}} = \left[ \tilde{y}_{1} ~ \tilde{y}_{2} ~ \cdots ~ \tilde{y}_{m} \right] = $ measured y-values \\

$\bf{e} = \left[ e_{1} ~ e_{2} ~ \cdots ~ e_{m} \right] = $ resisdual errors \\

$\bf{\hat{x}} = \left[ \hat{x}_{1} ~ \hat{x}_{2} ~ \cdots ~ \hat{x}_{m} \right] = $ estimated x-values \\

In similar way, Eq~\ref{eq:lse_w_error_matrix} and \ref{eq:lse_wo_error_matrix} can be represented using its compact matrix form as~\cite{crassidis2004dynamic}:

\begin{equation}
    \bf{\tilde{y}} = \bf{\it{H}} \bf{x} + \bf{v}
    \label{eq:lse_w_error_matrix}
\end{equation}

\begin{equation}
    \bf{\hat{y}} = \bf{\it{H}} \bf{\hat{x}}
    \label{eq:lse_wo_error_matrix}
\end{equation}

where \\

$\bf{x} = \left[ x_{1} ~ x_{2} ~ \cdots ~ x_{m} \right] = $ true x-values \\

$\bf{v} = \left[ v_{1} ~ v_{2} ~ \cdots ~ v_{m} \right] = $ measurements errors \\

$\bf{\hat{y}} = \left[ \hat{y}_{1} ~ \hat{y}_{2} ~ \cdots ~ \hat{y}_{m} \right] = $ estimated y-values \\

$\bf{\tilde{y}} = \left[ \tilde{y}_{1} ~ \tilde{y}_{2} ~ \cdots ~ \tilde{y}_{m} \right] = $ measured y-values \\

\subsubsection{Linear Least Squares}
The least squares principle selects particular $\hat{x}$ that minimizes the sum square of the residual errors as an optimum choice for the unknown parameters, given by~\cite{crassidis2004dynamic}:

\begin{equation}
    J = \frac{1}{2} e^{T} e
    \label{eq:lse}
\end{equation}

By substituting Eq~\ref{eq:lse_w_rerror} $e$ into the Eq~\ref{eq:lse}, applying the gradient of $\nabla_{\hat{x}} J$ and equaling to zero, the explicit solution for the optimal estimate is obtained:

\begin{equation}
    \bf{\hat{x}} = (\it{H}_{T} \it{H})^{-1} \it{H}_{T} \bf{\tilde{y}}
\end{equation}


\subsubsection{Weighted Least Squares}



\subsection{System Model}



\section{Methodology}
The simulation is conducted through Simulink, as depicted in Fig~\ref{fig:heat_system_diagram} the heating or control temperature system. This system comprises four primary components: the heating system, the $P_{0}$ computation, the $x_{0}$ calculation, and the least-squares filter implemented as a Matlab function.

\begin{figure}[H]
\centering
\includegraphics[width=1\linewidth]{figures/heat_system_diagram.png}
\caption{Diagram representing the complete system to be simulated}
~\label{fig:heat_system_diagram}
\end{figure}

The heating system is responsible for replicating the temperature variation, using an ambient temperature of 25$^{\circ}$C as a reference. There are two tested scenarios. In the first one, the simulation begins with a transition from 0$^{\circ}$C to 25$^{\circ}$C. Subsequently, the heating system activates, initiating a temperature alteration over a 150-second interval. Following this, the system deactivates and the simulation runs for an additional 150-second period. In the second scenario, there is a square input function that has a period of 20s and an amplitude of 100. The simulation runs for 400 seconds in order to have more data and see how the system improves over time. \\

The component that computes $P_{0}$ uses the Eq~\ref{eq:lse_recursion_P0}. The following Matlab code shows its implementation:

\lstinputlisting[language=Octave]{codes/compute_p0.m}

In the other hand, Eq~\ref{eq:lse_recursion_x0} is used to compute the value of $x_{0}$ and the implemented code is shown:

\lstinputlisting[language=Octave]{codes/compute_x0.m}

Finally, the least-squares estimation using the covariance recursion form based on Eq~\ref{eq:lse_recursion} is implemented as shown below:

\lstinputlisting[language=Octave]{codes/compute_pk1.m}

For the simulation, the following parameters are used:

\begin{itemize}
    \item $\tau$: the time constant of the system is set to 20s.

    \item $K_{p}$: the gain of the system is set to 0.5.

    \item $\sigma$: the standard deviation of the Gaussian white noise is set to 0.008.

    \item $\alpha$: is set to $10^{3}$

    \item $\beta$: is vector set to $\left[ 0.01 ~ 0.01 \right]$.

    \item $\Delta t$: the sampling time is set to 0.1 seconds.
\end{itemize}

and the code used is shown, as well as the computation of the parameters $a$, $b$, and $W$:

\lstinputlisting[language=Octave]{codes/constants.m}

where:

\begin{equation}
    a = \frac{-1}{\tau} = \frac{-1}{20} = -0.05
\end{equation}

\begin{equation}
    b = \frac{K_{p}}{\tau} = \frac{0.5}{20} = 0.025
\end{equation}

\begin{equation}
    W = \frac{1}{\sigma^{2}} = \frac{1}{0.08^{2}} = 156.25
\end{equation}

The system predicts the parameters $\phi$ and $\Gamma$ which can be computed using the Eq~\ref{eq:phi} and~\ref{eq:gamma}:

\begin{equation}
    \phi(k) = e^{a \cdot \Delta t} = e^{-0.05 \cdot 0.1} = 0.9950
\end{equation}

\begin{equation}
    \Gamma(k) = \frac{0.025}{-0.05} \left( e^{-0.05 \cdot 0.1} - 1\right) = 0.002494
\end{equation}


\section{Results and Discussion}



\begin{figure}[H]
    \centering
    \includegraphics[width=1\linewidth]{figures/simulink_system.png}
    \caption{}
    \label{fig:simulink_system}
\end{figure}



\begin{figure}[H]
    \centering
    \includegraphics[width=1\linewidth]{figures/simulink_plot.png}
    \caption{}
    \label{fig:simulink_plot}
\end{figure}



\begin{figure}[H]
    \centering
    \includegraphics[width=0.7\linewidth]{figures/simulink_params.png}
    \caption{}
    \label{fig:simulink_params}
\end{figure}



\begin{figure}[H]
    \centering
    \includegraphics[width=0.5\linewidth]{figures/simulink_params_gen.png}
    \caption{}
    \label{fig:simulink_params_gen}
\end{figure}



\section{Conclusions}
Heating and temperature control systems play a crucial role in enhancing both energy efficiency and occupant comfort. Accurately controlling indoor temperatures is indispensable across several fields, including but not limited to biotechnology, biology, food supply chain, transportation, automotive industries. These systems ensure optimal thermal comfort while providing a high degree of customization and flexibility to accommodate diverse user requirements and building characteristics. By minimizing energy wastage, they facilitate cost savings and contribute significantly to environmental sustainability efforts. \\

The least-squares filter exhibits remarkable robustness in handling data susceptible to random errors or fluctuations, showcasing its adaptability to diverse degrees of data complexity. While it excels in delivering accurate parameter estimations, particularly with large data, it does entail heightened computational demands as dataset size increases, necessitating careful resource allocation. Nevertheless, its versatility renders it invaluable across a wide range of fields and disciplines, underscoring its universal utility and relevance. \\

Simulink presents a robus platform for multidomain modeling and simulation, empowering engineers and researchers to design and simulate complex systems that span multiple domains. Its intuitive  graphical user interface and extensive library of pre-built blocks streamline the integration of diverse components and subsystems, facilitating accelerated prototyping and comprehensive system-level analysis and validation. With its user-friendly block diagram interface, users can quickly iterate on system designs, fine-tume parameters, and evaluate performance in real-time, thereby trimming development timelines and expenses. Furthermore, Simulink's seamless integration with MATLAB provides a cohesive environment for learning and applying computational techniques accros various engineering and scientific disciplines.


\bibliographystyle{IEEEtranN}
\bibliography{bibliography/bibliography}

\end{document}
